\documentclass[a4paper]{article}  % 定义文档类为A4纸的article(学术论文常用文档类)

% --- 页面布局设置 ---
\usepackage{geometry}  % 引入geometry宏包,用于调整页面边距等布局
\geometry{left=2.54cm, right=2.54cm, top=3.8cm, bottom=2.54cm, headheight=2.5cm, headsep=0.6cm}  % 设置左、右、上、下边距,页眉高度,页眉与正文间距

% --- pdflatex的字体和编码设置 ---
\usepackage[utf8]{inputenc}  % 设置输入编码为UTF-8,支持中文字符输入
\usepackage[T1]{fontenc}     % 设置输出字体编码为T1,增强字体兼容性
\usepackage{newtxtext}        % 引入newtxtext宏包,使用Times风格的文本字体
\usepackage{newtxmath}        % 引入newtxmath宏包,使用Times风格的数学字体

% --- 数学、插图、表格相关宏包 ---
\usepackage{amsmath}   % 引入amsmath宏包,提供丰富的数学公式排版功能
\usepackage{graphicx}  % 引入graphicx宏包,用于插入图片
\usepackage{float}     % 引入float宏包,提供[H]选项让图表强制固定位置
\usepackage{booktabs}  % 引入booktabs宏包,用于绘制专业的三线表
\usepackage{caption}   % 引入caption宏包,自定义图表标题格式
\usepackage{tabularx}  % 引入tabularx宏包,支持自动列宽的表格

% --- 格式设置宏包 ---
\usepackage{xcolor}    % 引入xcolor宏包,支持颜色定义和使用
\usepackage{titlesec}  % 引入titlesec宏包,用于自定义章节标题格式
\usepackage{setspace}  % 引入setspace宏包,用于调整行间距
\usepackage{indentfirst} % 引入indentfirst宏包,让章节后的第一段也缩进
\usepackage{multicol}  % 引入multicol宏包,用于实现多栏排版

% --- 颜色定义 ---
\definecolor{myred}{HTML}{C71D31}  % 定义名为myred的颜色,HTML色号为C71D31

% --- 页眉页脚设置 ---
\usepackage{fancyhdr}  % 引入fancyhdr宏包,用于自定义页眉页脚
\pagestyle{fancy}      % 设置页面样式为fancy风格
\fancyhf{}             % 清空当前页眉页脚的所有内容
\fancyhead[C]{\includegraphics[width=\textwidth]{header.png}} % 在页眉中间插入宽度为文本宽度的header.png图片
\renewcommand{\headrulewidth}{0pt} % 将页眉下方的横线宽度设为0(即去掉横线)
\fancyfoot[C]{\thepage} % 在页脚中间插入当前页码

% --- 字体大小定义 ---
\renewcommand{\normalsize}{\fontsize{10.5pt}{10.5pt}\selectfont} % 重新定义normalsize为10.5pt字号,10.5pt行间距
\setlength{\parskip}{0pt} % 段落之间的垂直间距设为0pt
\singlespacing             % 设置全文为单倍行间距

% 一级标题(section)格式设置
\titleformat{\section}
{\color{myred}\bfseries\fontsize{12pt}{12pt}\selectfont} % 标题格式:myred颜色、加粗、12pt字号/行间距
{\thesection.} % 标题编号格式(如“1.”)
{0.5em}        % 编号与标题文字的间距
{}             % 标题前的额外代码(此处无)
\titlespacing*{\section}{0pt}{0.9\baselineskip}{1.0ex plus .2ex} % 标题间距:左缩进0pt,上方0.9倍行间距,下方1.0ex(带弹性)

% 二级标题(subsection)格式设置
\titleformat{\subsection}
{\color{myred}\bfseries\fontsize{12pt}{12pt}\selectfont} % 同section格式:myred、加粗、12pt
{\thesubsection} % 标题编号格式(如“1.1”)
{0.5em}          % 编号与文字间距
{}               % 标题前额外代码
\titlespacing*{\subsection}{0pt}{1.0ex plus .2ex minus .2ex}{1.0ex plus .2ex} % 标题间距:左0pt,上方1.0ex(弹性),下方1.0ex(弹性)

% --- 文档开始 ---
\begin{document}
	
	% ==========================================
	% 【模板:标题、作者、单位】
	% ==========================================
	\begin{flushleft}  % 开始左对齐环境
		\interlinepenalty=10000 % 防止标题处换行分页
		% 论文标题
		{\fontsize{18pt}{18pt}\selectfont \bfseries 
			Your Shit Title Here  % 此处替换为你的论文标题
			\par}
		
		% 作者与单位(通用占位符)
		\vspace{1\baselineskip} % 插入1倍行间距的垂直空白
		{\fontsize{12pt}{12pt}\selectfont 
			First Author$^{1,*}$, Second Author$^{2}$ \par  % 作者名,上标1、2为单位序号,*为通讯作者标记
			\vspace{0.2\baselineskip} % 小垂直间距
			$^{1}$Department/Institution, University, City 100000, China \par % 第一作者单位:系/机构、大学、城市、邮编、国家
			$^{2}$Department/Institution, University, City 100000, China \par % 第二作者单位
			\vspace{0.2\baselineskip} % 小垂直间距
		}
		% 中间图片
		\vspace{0.6\baselineskip} % 垂直间距
		\noindent\includegraphics[width=\textwidth]{mider.png} % 插入宽度为文本宽度的mider.png图片,无缩进
		\vspace{0.6\baselineskip} % 垂直间距
	\end{flushleft}  % 结束左对齐环境
	
	% ==========================================
	% 【模板:摘要】
	% ==========================================
	\begingroup  % 开始一个局部组,内部设置仅在组内生效
	\noindent {\color{myred}\fontsize{15pt}{15pt}\selectfont \bfseries Abstract} % 摘要标题:无缩进、myred色、15pt、加粗
	\vspace{0.5\baselineskip} % 垂直间距
	\endgroup    % 结束局部组
	
	\begingroup  % 开始局部组
	\fontsize{12pt}{12pt}\selectfont % 12pt字号/行间距
	\setlength{\parindent}{2em}       % 段落缩进设为2em
	\vspace{0.5\baselineskip}         % 垂直间距
	
	Abstract content here. Describe the background, methods, results, and conclusions of your research. % 此处替换为摘要内容,描述研究背景、方法、结果、结论
	
	\vspace{1\baselineskip} % 垂直间距
	\endgroup    % 结束局部组
	
	% ==========================================
	% 【模板:关键词】
	% ==========================================
	\begingroup  % 开始局部组
	\noindent {\fontsize{12pt}{12pt}\selectfont \bfseries Keywords:} % 关键词标签:无缩进、12pt、加粗
	{\fontsize{12pt}{12pt}\selectfont 
		Keyword 1; Keyword 2; Keyword 3; Keyword 4} % 此处替换为关键词,用分号分隔
	\vspace{1\baselineskip} % 垂直间距
	\endgroup    % 结束局部组
	
	\vspace{0.6\baselineskip} % 垂直间距
	\noindent\includegraphics[width=\textwidth]{footer.png} % 插入宽度为文本宽度的footer.png图片,无缩进
	\vspace{0.6\baselineskip} % 垂直间距
	
	% ==========================================
	% 正文主体(双栏排版)
	% ==========================================
	\begin{multicols}{2} % 开始双栏环境
		
		\section{Introduction} % 一级标题“引言”
		
		Introduction content here. Explain the research background, problems to be solved, and the significance of this work. % 此处替换为引言内容,解释研究背景、要解决的问题、工作意义
		
		\section{Theoretical Model} % 一级标题“理论模型”
		
		\subsection{Example Model} % 二级标题“示例模型”
		We describe the system using the time-independent Schrödinger equation: % 此处为正文内容:我们用定态薛定谔方程描述系统
		
	\end{multicols} % 临时结束双栏环境(为了让公式通栏显示)
	\begin{equation} % 开始单行公式环境
		-\frac{\hbar^2}{2m} \frac{\partial^2 \psi}{\partial x^2} + V(x)\psi = E \psi % 定态薛定谔方程公式
	\end{equation}
	
	\begin{figure}[H] % 开始图片环境,[H]强制固定位置
		\centering % 图片居中
		\includegraphics[width=0.2\textwidth]{joker.png} % 插入宽度为0.2倍文本宽度的joker.png
		\caption{Joker.} % 图片标题
		\label{fig:joker} % 图片标签,用于文中引用
	\end{figure}
	
	\begin{table}[H] % 开始表格环境,[H]强制固定位置
		\centering % 表格居中
		\caption{shit} % 表格标题
		\label{tab:shit} % 表格标签,用于文中引用
		\begin{tabularx}{\textwidth}{>{\centering\arraybackslash}X>{\centering\arraybackslash}X>{\centering\arraybackslash}X>{\centering\arraybackslash}X>{\centering\arraybackslash}X} % tabularx表格,宽度为文本宽度,5列自动居中的X列
			\toprule % 顶部粗线(booktabs三线表)
			\textbf{shit} & \textbf{shit} & \textbf{shit} & \textbf{shit} & \textbf{shit} \\ % 表头,加粗
			\midrule % 中间细线(booktabs三线表)
			shit & shit & shit & shit & shit \\ % 表格内容行
			shit & shit & shit & shit & shit \\ % 表格内容行
			shit & shit & shit & shit & shit \\ % 表格内容行
			\bottomrule % 底部粗线(booktabs三线表)
		\end{tabularx}
	\end{table}
	
	
	\begin{multicols}{2} % 重新开始双栏环境
		
		\noindent % 无缩进
		Here, $T$ is the kinetic energy operator, and $V(x)$ is the potential energy. % 正文内容:这里$T$是动能算符,$V(x)$是势能
		
		\section{Experimental Results} % 一级标题“实验结果”
		
		\section{Conclusion} % 一级标题“结论”
		
		Conclusion content here. Summarize your main findings and contributions. % 此处替换为结论内容,总结主要发现和贡献
		
		\begin{thebibliography}{99} % 开始参考文献环境,99表示最多支持两位数的参考文献编号
			
			\bibitem{example_ref1} % 参考文献1的标签,用于文中引用
			Author, A., \& Author, B. (Year). % 作者、年份
			\textit{Book title}. % 书名(斜体)
			Publisher. % 出版社
			
			\bibitem{example_ref2} % 参考文献2的标签
			Author, C., et al. (Year). % 作者(等)、年份
			Article title. % 文章标题
			\textit{Journal Name}, \textit{Volume}(Issue), Pages. % 期刊名(斜体)、卷(期)、页码
			
		\end{thebibliography}
		
	\end{multicols} % 结束双栏环境
	
\end{document} % 文档结束